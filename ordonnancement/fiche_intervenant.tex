\documentclass[a4paper]{article}

%% Language and font encodings
\usepackage[french]{babel}
\usepackage[utf8x]{inputenc}
\usepackage[T1]{fontenc}
\usepackage{wrapfig}

%% Sets page size and margins
\usepackage[a4paper,top=3cm,bottom=2cm,left=3cm,right=3cm,marginparwidth=1.75cm]{geometry}

%% Useful packages
\usepackage{amsmath}
\usepackage{graphicx}
\usepackage[colorinlistoftodos]{todonotes}
\usepackage[colorlinks=true, allcolors=blue, breaklinks=true]{hyperref}

\newcommand{\guill}[1]{\og{}#1\fg{}}

\title{Le planning du cuisinier}
\author{Adrien Krähenbühl \and Basile Sauvage \and Julien Narboux}

\begin{document}
\maketitle

\begin{abstract}
Description d'une activité d'informatique débranchée sur le thème de l'ordonnancement.
Cette activité a été préparée pour la Fête de la Science 2018 à Strasbourg.
\end{abstract}


\section{Contexte}

Un cuisinier se retrouve avec un ensemble de plats à préparer qui nécessitent chacun un certain temps de préparation et un certain temps de cuisson. Le cuisinier ne peut préparer qu'un plat à la fois, de même que le four ne peut accueillir qu'un plat à la fois.

L'objectif est de trouver l'ordre de préparation et de cuisson qui permet de terminer l'ensemble des plats le plus rapidement possible.

\section{Matériel}

La matériel consiste en un diagramme de Gantt manipulable. Il peut être construit en bois pour une utilisation intensive ou simplement en papier pour une activité en classe.

Les temps de cuisson et de préparation des plats sont matérialisés par des rectangles de longueur proportionelle.




\section{Défis}

Nous proposons quatre défis qui sont conçus de la manière suivante:
\begin{enumerate}
\item Un premier défi (croissant, brioche, pain) dont la solution optimale consiste à trier les plats par ordre de temps de préparation croissante.
\item Un deuxième défi (tarte flambée, jambonneau, bretzel) qui montre que trier par temps de préparation croissant n'est pas toujours optimal, mais le trie par temps de cuisson décroissant fonctionne.
\item Un troisième défi (poulet, gâteau, poisson) dont la solution optimale n'est ni un tri croissant des temps de préparation, ni un tri décroissant des temps de cuisson.
\item Un quatrième défi (9 plats) permettant de tester l'algorithme de Johnson.
\end{enumerate}





\end{document}