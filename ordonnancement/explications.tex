\documentclass[landscape]{beamer}

\usepackage[utf8]{inputenc}
\usepackage[french]{babel}

\usepackage{tikz}

\title{Le jeu du cuisinier}
\date{}

\setbeamertemplate{navigation symbols}{} 
%\usetheme{Dresden}
\usetheme{Madrid}
%\usecolortheme{seahorse}
\usecolortheme{rose}

\begin{document}

%\maketitle

%\setbeamertemplate{footline}[page number]

\begin{frame}{Le jeu du cuisinier}

\begin{block}{Le but du jeu}

\begin{itemize}
\item Un cuisinier se retrouve avec un ensemble de plats à préparer qui nécessitent chacun un certain temps de préparation \includegraphics[width=0.5cm]{icons/toque.png} et un certain temps de cuisson \includegraphics[width=0.5cm]{icons/four.png}. 

\item
Le cuisinier ne peut préparer qu'\textbf{un plat à la fois}, de même que le four ne peut accueillir qu'un plat à la fois.

\item
L'objectif est de trouver l'ordre de préparation et de cuisson qui permet de terminer l'ensemble des plats le plus rapidement possible.
\end{itemize}

\end{block}

\end{frame}


\begin{frame}{Une solution}

%Principe : on détermine un ordre sur les plats, chacun ayant deux tâches, préparation et cuisson.


\begin{block}{Algorithme de Johnson:}

\begin{enumerate}
\item Sélectionner la tâche la plus courte.
\item Si c'est un temps de préparation, placer ce plat au début du planning. Sinon, c'est un temps de cuisson, placer ce plat en fin de planning.
\item Continuer avec le reste des plats de la même manière en remplissant le planning par le centre.
\end{enumerate}

\end{block}


\end{frame}


\end{document}
